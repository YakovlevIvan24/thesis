\documentclass[a4paper,12pt]{article}
\usepackage{float}
\usepackage{gensymb}
\usepackage[left=2cm,right=2cm,top=2cm,bottom=2cm]{geometry}
\usepackage{cmap}					% поиск в PDF
\usepackage[T2A]{fontenc}			% кодировка
\usepackage[utf8]{inputenc}			% кодировка исходного текста
\usepackage[english,russian]{babel}	% локализация и переносы
\usepackage{amsmath,amsfonts,amssymb,amsthm,mathtools}
\usepackage[warn]{mathtext}
\usepackage{graphicx}
\usepackage{xcolor}
\usepackage{adjustbox}
\usepackage{soul}
\usepackage{amssymb}
\usepackage{listings}
% \usepackage{bbold}
% \usepackage{dsfont}
% \usepackage{amsfonts}
\setcounter{MaxMatrixCols}{15}
\graphicspath{{C:\Users\123\Desktop}}
\begin{document}
\begin{titlepage}
	\centering
	\vspace{5cm}
	{\scshape\LARGE Московский физико-технический институт \par}
	\vspace{4cm}
	{\scshape\Large Черновик диплома \par}
	\vspace{1cm}
        
           	{\huge\bfseries  Численное моделирование воздействия вибрационной нагрузки на образец в трёхмерном случае. \par} 
        

	\vspace{1cm}
	\vfill
\begin{flushright}
	{\Large }\par
	\vspace{0.3cm}
	{\large  \par
                 
 } \par

\end{flushright}
	

	\vfill

% Bottom of the page
	Долгопрудный, 2021 г.
\end{titlepage}


\section{Постановка задачи}
\textcolor{red}{todo:
1. Написать объяснения преобразований тензоров} \par
Есть пластина, она занимает объем $\overline{\Omega} \times [-\frac{1}{2}h, \frac{1}{2}h] \subset \mathbb{R}^3$, h - толщина пластины. $\Omega$ - срединная плоскость. Граница пластины состоит из двух частей:
\begin{equation}
    \partial \Omega = \text{Г}_{\text{с}} \cup \text{Г}_{\text{f}}
\end{equation}
где $\text{Г}_{\text{с}}$ - закрепленный конец, $\text{Г}_f$ - свободный конец
\begin{figure}[H]
	\begin{center}
		\includegraphics[width = 0.7\textwidth]{tp3d.png}
		\caption{Модель пластины в 3D}
	\end{center}
\end{figure}


уравнение динамики для пластины: 
\begin{equation}
\rho\ddot{\overline{u}} -  \nabla \cdot ( \textbf{C:e}) - \overline{Q} = 0,
\end{equation}

\textcolor{red}{В Зинкевиче сделано примерно все то же самое но в скалярном случае (стр. 141, стр.469)}
\begin{figure}[H]
    \includegraphics[]{Zinkevich2.png}
    \caption{Аналог в Зинкевиче}
    \label{fig:enter-label}
\end{figure}
\textcolor{red}{Как это перенести для этого векторного уравнения - мне не очевидно, поэтому я вывел уравнения в тензорном виде} \par
где $\nabla \cdot$ - дивергенция ($\nabla \otimes $ - градиент), $\rho$ - плотность материала, $\overline{Q}$ - внешняя нагрузка, $\textbf{C}$ - тензор уругих модулей: \textcolor{red}{Не уверен, что он верно записан}
\begin{equation}
\textbf{C} = \Lambda
\begin{pmatrix}
    1-\nu & \nu & \nu &  0 & 0 &0 &0 &0 & 0 \\
    \nu & 1-\nu & \nu & 0 & 0 & 0 & 0&0 &0 \\
    \nu & \nu & 1-\nu & 0&0&0&0&0&0\\
    0 & 0 & 0 & 1+\nu & 1+\nu & 0 &0 &0&0 \\
    0 & 0 & 0 & 1+\nu & 1+\nu & 0 &0 &0&0 \\
    0&0 &0 &0 &0 &1+\nu &1+\nu &0 &0 \\
    0&0 &0 &0 &0 &1+\nu &1+\nu &0 &0 \\
    0&0 &0 &0 &0 &0 &0 &1+\nu &1+\nu \\
    0&0 &0 &0 &0 &0 &0 &1+\nu &1+\nu \\
    
\end{pmatrix}
\end{equation}
где $\Lambda = \frac{E}{(1+\nu)(1-2\nu)}$, \textbf{e} - тензор малых деформаций:
\begin{equation}
    \textbf{e} = \frac{1}{2}(\nabla \otimes \overline{u} + \nabla \otimes \overline{u}^T)
\end{equation}
\section{Метод}
Задачу можно свести к задаче о минимизации функционала:

% \begin{center}
        
\begin{equation}
\chi = \int_V f(x,y,z,\overline{u}, \nabla \otimes \overline{u})dxdydz  \rightarrow min \Longleftrightarrow -\frac{\partial f}{\partial \overline{u}} + \nabla \cdot \frac{\partial f}{\partial (\nabla \otimes \overline{u} )} = 0
\end{equation}
\textcolor{red}{ Наверное, $\frac{\partial f}{\partial \overline{u}} = \nabla_{\overline{u}} \otimes f$}
Последнее уравнение - уравнение Эйлера-Лагранжа. 
\par
Рассмотрим сначала уравнение (2) без первого слагаемого.
Внимательно посмотрев на это уравнение, получаем
\begin{equation}
    \frac{\partial f}{\partial (\nabla \otimes \overline{u} )} = \frac{1}{2}\textbf{C}:(\nabla \otimes \overline{u} + \nabla \otimes \overline{u}^T)
\end{equation}
Заметим, что
\begin{equation}
    \frac{d (\textbf{A}:\textbf{A}^T)}{d\textbf{A}} = \textbf{A} + \textbf{A}^T
\end{equation}
Отсюда следует, что
\begin{equation}
    f = \frac{1}{2}\textbf{C}:\nabla \otimes \overline{u} : \nabla \otimes \overline{u}^T - \overline{Q}\cdot \overline{u}
\end{equation}
\textcolor{red}{Я не слишком глубоко проникся тензорным анализом (возникают два вопроса: 1. $\partial \leftrightarrow d$. 2. что с константой интегрирования), поэтому выше, ориентируясь на Зинкевича я выписываю интуитивно ожидаемый результат}
Рассмотрим тетраэдральный элемент. Поле перемещений аппроксимируем линейной функцией координат:
\begin{equation}
    \overline{u} = \alpha_1 + \alpha_2 x + \alpha_3 y + \alpha_4 z 
\end{equation}
Тогда для вектора перемещений в узле i:
\begin{equation}
    \overline{u}_i = \alpha_1 + \alpha_2 x_i + \alpha_3 y_i + \alpha_4 z_i
\end{equation}
Отсюда можно записать
\begin{equation}
\begin{split}
    \overline{u} = \frac{1}{6V}[(a_i + b_i x + c_i y + d_i z)\overline{u}_i + (a_j + b_j x + c_j y + d_j z)\overline{u}_j \\  + (a_m + b_m x + c_m y + d_m z)\overline{u}_m  + (a_p + b_p x + c_p y + d_p z)\overline{u}_p]
\end{split}
\end{equation}
где
\begin{equation}
    V = det
    \begin{vmatrix}
1 & x_i & y_i & z_i \\ 
1 & x_j & y_j & z_j \\ 
1 & x_m & y_m & z_m \\ 
1 & x_p & y_p & z_p \\
    \end{vmatrix}
\end{equation}
коэффициенты $a_i, b_i, c_i, d_i$ определяются как
\begin{align}
    a_i =    det 
    \begin{vmatrix}
 x_j & y_j & z_j \\ 
x_m & y_m & z_m \\ 
x_p & y_p & z_p \\
    \end{vmatrix},
\:\: b_i  = -det 
    \begin{vmatrix} 
1 & y_j & z_j \\ 
1 & y_m & z_m \\ 
1 & y_p & z_p \\
    \end{vmatrix}
\:\: c_i =     -det \begin{vmatrix}
 x_j & 1 & z_j \\ 
x_m & 1 & z_m \\ 
x_p & 1 & z_p \\
    \end{vmatrix},
\:\: d_i =     -det\begin{vmatrix}
 x_j & y_j & 1 \\ 
x_m & y_m & 1 \\ 
x_p & y_p & 1 \\
    \end{vmatrix}
\end{align}
Перемещение произвольной точки можно записать в виде 
\begin{equation}
    \overline{u} = [\mathbb{I} N_i,\mathbb{I} N_j,\mathbb{I} N_m,\mathbb{I} N_p] \cdot \overline{u}^e = [\textbf{N}] \overline{u}^e
\end{equation}
где $N_i = \frac{a_i + b_i x + c_i y + d_i z}{6V}, \mathbb{I} - $ единичная матрица, $\overline{u}^e = (u_1, v_1,\omega_1, u_2, v_2,  \omega_2 ...\omega_4) $ - вектор размера (12,1) неизвестных перемещений для элемента. \par
Рассмотрим данный функционал на элементе e: $\chi \rightarrow \chi^e$:

\begin{figure}[H]
    \centering
    \includegraphics[width=0.5\linewidth]{tetrahetron.png}
    \caption{Тетраэдральный элемент}
    \label{fig:enter-label}
\end{figure}

И возьмем производную по $\overline{u}_i$ - вектору перемещения в узле i:

 
\begin{align}
\nabla_{\overline{u}_i} \otimes \chi^e = \int_{V^e} \Bigg(\frac{1}{2} \nabla_{\overline{u}_i} \otimes  \nabla \otimes \overline{u} : (\nabla \otimes \overline{u}^T:\textbf{C}) + (\nabla_{\overline{u}_i} \otimes  \nabla \otimes \overline{u})^T : (\textbf{C}:\nabla \otimes \overline{u})  - \\ - \nabla_{\overline{u}_i} \otimes\overline{u} \cdot \overline{Q}\Bigg) dV
\end{align}
Учтем, что 
\begin{equation}
\begin{split}
    \nabla \otimes \overline{u} = (\nabla \otimes [\textbf{N}]) \cdot \overline{u}^e, \\
    \nabla_{\overline{u}_i} \otimes \nabla \otimes \overline{u}  = (\nabla \otimes [\textbf{N}]) \cdot (\nabla_{\overline{u}_i} \otimes \overline{u}^e)^T
\end{split}
\end{equation}

Тогда (15) можно записать как
\begin{equation}
    \begin{split}
        \nabla_{\overline{u}_i} \otimes \chi^e   =  \int_{V^e} \Bigg(\frac{1}{2} \nabla \otimes [\textbf{N}] \cdot (\nabla_{\overline{u_i}} \otimes \overline{u}^e)^T:((\nabla \otimes[\textbf{N}] \cdot \overline{u}^e)^T :\textbf{C}) + \\
        + \frac{1}{2} (\nabla \otimes [\textbf{N}] \cdot (\nabla_{\overline{u_i}} \otimes \overline{u}^e)^T)^T:\textbf{C}:\nabla \otimes [\textbf{N}] \cdot \overline{u}^e - \nabla_{\overline{u}_i} \otimes\overline{u} \cdot \overline{Q}\Bigg) \space dV = \\
        =\int_{V^e} \Bigg( \bigg\{  \frac{1}{2}\nabla \otimes [\textbf{N}] \cdot (\nabla_{\overline{u_i}} \otimes \overline{u}^e)^T:((\nabla \otimes[\textbf{N}])^T:\textbf{C})^T + \\ 
        +(\nabla \otimes [\textbf{N}] \cdot (\nabla_{\overline{u_i}} \otimes \overline{u}^e)^T)^T:\textbf{C}:\nabla \otimes [\textbf{N}] \bigg\} \cdot   \overline{u}^e  - \nabla_{\overline{u}_i} \otimes\overline{u} \cdot \overline{Q} \Bigg)dV
    \end{split}
\end{equation}

Рассмотрим это выражение для узла j:
\begin{equation}
    \begin{split}
        \nabla_{\overline{u}_i} \otimes \chi^e_j = \int_{V^e} \Bigg(\frac{1}{2}\bigg\{ \nabla \otimes [\textbf{N}] \cdot (\nabla_{\overline{u_i}} \otimes \overline{u}^e)^T:((\nabla \otimes[\textbf{N}]_j)^T:\textbf{C})^T + \\ 
        +(\nabla \otimes [\textbf{N}] \cdot (\nabla_{\overline{u_i}} \otimes \overline{u}^e)^T)^T:\textbf{C}:\nabla \otimes [\textbf{N}]_j \cdot \bigg\} \overline{u}^e_j  - \nabla_{\overline{u}_i} \otimes\overline{u}_j \cdot \overline{Q}_j \Bigg)\space dV
    \end{split}
\end{equation}
Выражение в фигурных скобках - \textbf{матрица 3x3}. \textcolor{red}{Здесь, возможно, не очевидный момент - i,j = {1,2,3,4} - нумеруют узел в тетраэдральном элементе, это никак не связано с тем, что в скобках матрица. Дело в том, что $\overline{u}$ - 3x1 вектор, поэтому связь между i-м и j-м узлами теперь задается матрицей 3x3. Поэтому в разделе пояснений к коду, (и пояснений к преобразованиям тензоров, (когда они появятся) )при переходе к индексной записи тензоров я использую греческие буквы, чтобы они не пересекались с i и j}
Тогда для всего элемента
\begin{equation}
    \nabla_{\overline{u}_i} \otimes \chi^e_j = h^e \cdot \overline{u}^e + \overline{F}^e
\end{equation},
где
\begin{equation}
    h_{ij}^e = \frac{1}{2}\int_{V^e} \Bigg\{ \nabla \otimes [\textbf{N}] \cdot (\nabla_{\overline{u_i}} \otimes \overline{u}^e)^T:((\nabla \otimes[\textbf{N}]_j)^T:\textbf{C})^T + \\ 
        (\nabla \otimes [\textbf{N}] \cdot (\nabla_{\overline{u_i}} \otimes \overline{u}^e)^T)^T:\textbf{C}:\nabla \otimes [\textbf{N}]_j \Bigg\} dV
\end{equation}
\begin{equation}
    \overline{F}^e_i = - \int_{V^e}\nabla_{\overline{u}_i} \otimes\overline{u}_j \cdot \overline{Q}_j
\end{equation}

Минимизирующая система для всех элементов:

\begin{equation}
    \nabla_{\overline{u}} \otimes \chi = 0 =[H] \cdot \overline{u} +  \overline{F}
\end{equation}
где 
\begin{equation}
    H_{ij} = \sum h_{ij}^e  \hspace{1cm} F_i = \sum F_i^e
\end{equation}


Аналогично получается матрица масс для первого слагаемого в уравнении (2):
\textcolor{red}{Ее я не выводил руками, а написал по аналогии}
\begin{equation}
    M_{ij}^e = \int_{V^e} [\textbf{N}]_i^T \rho [\textbf{N}]_j dV
\end{equation}

Тогда итоговое уравнение выглядит следующим образом:
\begin{equation}
    [H]\overline{u} + [M] \ddot{\overline{u}} + \overline{F} = 0
\end{equation}
\section{Пояснения к коду plate3d.py}
Вернемся к страшной формуле для $h_{ij}$ чтобы показать как она выражается в коде. 
\begin{align}
& \nabla \otimes [\textbf{N}]_{0\alpha \beta} = \frac{\partial}{\partial x}
\begin{pmatrix}
N_1 & 0 & 0 & N_2 & 0 & 0 & N_3 & 0 & 0 & N_4 & 0 & 0 \\
0 & N_1 & 0 & 0 & N_2 & 0 & 0 & N_3 & 0 & 0 & N_4 & 0 \\
0 & 0 & N_1 & 0 & 0 & N_2 & 0 & 0 & N_3 & 0 & 0 & N_4
\end{pmatrix} = \\
&=\begin{pmatrix}
b_1 & 0   & 0   & b_2 & 0   & 0   & b_3 & 0   & 0   & b_4 & 0   & 0 \\
0   & b_1 & 0   & 0   & b_2 & 0   & 0   & b_3 & 0   & 0   & b_4 & 0 \\
0   & 0   & b_1 & 0   & 0   & b_2 & 0   & 0   & b_3 & 0   & 0   & b_4\\
\end{pmatrix}
\end{align}
Для $\nabla \otimes [\textbf{N}]_{1\alpha \beta}$ и $\nabla \otimes [\textbf{N}]_{2\alpha \beta}$ соответственно $\frac{\partial}{\partial y}$ и $\frac{\partial}{\partial z}$, в матрице $c_i$ и $d_i $. 
\begin{align}
& \nabla_{\overline{u_i}} \otimes \overline{u}^e =\begin{pmatrix}
    \frac{\partial}{\partial u_i} u_1 & \frac{\partial}{\partial u_i} v_1 & \frac{\partial}{\partial u_i} \omega_1 &\frac{\partial}{\partial u_i} u_2 & \frac{\partial}{\partial u_i} v_2 & \frac{\partial}{\partial u_i} \omega_2 &\frac{\partial}{\partial u_i} u_3 & \frac{\partial}{\partial u_i} v_3 & \frac{\partial}{\partial u_i} \omega_3 &\frac{\partial}{\partial u_i} u_4 & \frac{\partial}{\partial u_i} v_4 & \frac{\partial}{\partial u_i} \omega_4 \\
    \frac{\partial}{\partial v_i} u_1 & \frac{\partial}{\partial v_i} v_1 & \frac{\partial}{\partial v_i} \omega_1 &\frac{\partial}{\partial v_i} u_2 & \frac{\partial}{\partial v_i} v_2 & \frac{\partial}{\partial v_i} \omega_2 &\frac{\partial}{\partial v_i} u_3 & \frac{\partial}{\partial v_i} v_3 & \frac{\partial}{\partial v_i} \omega_3 &\frac{\partial}{\partial v_i} u_4 & \frac{\partial}{\partial v_i} v_4 & \frac{\partial}{\partial v_i} \omega_4 \\
    \frac{\partial}{\partial \omega_i} u_1 & \frac{\partial}{\partial \omega_i} v_1 & \frac{\partial}{\partial \omega_i} \omega_1 &\frac{\partial}{\partial \omega_i} u_2 & \frac{\partial}{\partial \omega_i} v_2 & \frac{\partial}{\partial \omega_i} \omega_2 &\frac{\partial}{\partial \omega_i} u_3 & \frac{\partial}{\partial \omega_i} v_3 & \frac{\partial}{\partial \omega_i} \omega_3 &\frac{\partial}{\partial \omega_i} u_4 & \frac{\partial}{\partial \omega_i} v_4 & \frac{\partial}{\partial \omega_i} \omega_4\\
    \end{pmatrix}^T \\
    = &[0,...\mathbb{I}_i,..0]
\end{align}
    


Поэтому удобнее их воспринимать вместе:
\begin{equation}
    \nabla \otimes [\textbf{N}] \cdot (\nabla_{\overline{u_i}} \otimes \overline{u}^e)^T = \nabla \otimes [\textbf{N}]_{\alpha \beta\gamma} \cdot (\nabla_{\overline{u_i}} \otimes \overline{u}^e)^T_{\gamma \zeta} = \textbf{B}_{\alpha \beta \zeta}
\end{equation}
\textcolor{red}{Здесь сидит индекс i - т.к. эта матрица это расписанный $\nabla_{\overline{u}_i} \otimes  \nabla \otimes \overline{u}$}
\begin{equation}
    \textbf{B}_{0\beta \zeta} = 
\begin{pmatrix}
b_i & 0   & 0   \\
0   & b_i & 0   \\
0   & 0   & b_i \\
\end{pmatrix},  \textbf{B}_{1\beta \zeta} = 
\begin{pmatrix}
c_i & 0   & 0   \\
0   & c_i & 0   \\
0   & 0   & c_i \\
\end{pmatrix}, \textbf{B}_{2\beta\zeta} = 
\begin{pmatrix}
d_i & 0   & 0   \\
0   & d_i & 0   \\
0   & 0   & d_i \\
\end{pmatrix}
\end{equation}
Аналогично для 
\begin{equation}
    (\nabla \otimes[\textbf{N}]_j)_{\alpha \beta \gamma}^T:\textbf{C}_{\beta \gamma \phi \zeta}) = \textbf{A}_{\alpha \phi \zeta}
\end{equation}
\textcolor{red}{ Здесь сидит индекс j т.к мы рассмотрели крокодила (15) на узле j}
Для матрицы масс: \par
\begin{align}
    & N_i N_j= \frac{a_i + b_i x + c_i y + d_i z}{6V} \cdot  \frac{a_j + b_j x + c_j y + d_j z}{6V} = \\
    &\frac{a_i + b_i (x_b + \tilde{x}) + c_i (y_b + \tilde{y}) + d_i (z_b + \tilde{z})}{6V} \cdot  \frac{a_j + b_j (x_b + \tilde{x}) + c_j (y_b + \tilde{y}) + d_j (z_b + \tilde{z})}{6V}
\end{align}
где $x_b,y_b,z_b$ -координаты барицентра
В системе барицентра
\begin{equation}
    \int dxdydz = V = \frac{1}{6}     \begin{vmatrix}
1 & x_i & y_i & z_i \\ 
1 & x_j & y_j & z_j \\ 
1 & x_m & y_m & z_m \\ 
1 & x_p & y_p & z_p \\
    \end{vmatrix} 
\end{equation}
\textcolor{red}{В этот момент понял, что возможно ошибся в коэффициентах при вычислении матрицы масс}

\begin{equation}
    \int xdxdydz = \int ydxdydz = 0 
\end{equation}
\begin{equation}
    \int x^2 dxdydz =  \frac{V}{10} (\sum_i x^2_i + \sum_{ij, i<j} x_i x_j) 
\end{equation}
\begin{equation}
    \int xy \space dxdydz =  \frac{V}{20} (\sum_i x_i y_i + \sum_{ij, i<j}( x_i y_j + x_j y_i)) 
\end{equation}
\textcolor{red}{Формулы выше расходятся с Зинкевичем из-за перекрестных слагаемых, deepseek утверждает что у Зинкевича формулы для интегрирования по равностороннему треугольнику, я ему верю, закодить эти формулы мне показалось легче, чем закодить преобразование треугольника}
\section{Что получилось}

Ищем собственные значения:
\begin{equation}
    u(x,y,z,t) = u(x,y,z)\cdot e^{i\omega t}
\end{equation}
Тут я заходил с двух сторон: \par
1.Как в Зинкевиче рассмотреть эту систему как задачу на собственные значения и вычислить $\omega$ из уравнения $det([H] - \omega^2 [M]) = 0$ \par
2. Задать $\omega$ и решить систему $([H] - \omega^2[M])\overline{u} + \overline{F} = 0$. \par
В обоих случаях пока осмысленного ничего не получил, код пока сильно не дебажил \par
Проблема, как мне кажется, в том, что в $\textbf{C}$ огромные значения - из-за этого со СЛАУ возникают проблемы.
\end{document}